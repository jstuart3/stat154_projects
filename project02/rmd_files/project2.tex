\documentclass[11pt]{article}\usepackage[]{graphicx}\usepackage[]{color}
%% maxwidth is the original width if it is less than linewidth
%% otherwise use linewidth (to make sure the graphics do not exceed the margin)
\makeatletter
\def\maxwidth{ %
  \ifdim\Gin@nat@width>\linewidth
    \linewidth
  \else
    \Gin@nat@width
  \fi
}
\makeatother

\definecolor{fgcolor}{rgb}{0.345, 0.345, 0.345}
\newcommand{\hlnum}[1]{\textcolor[rgb]{0.686,0.059,0.569}{#1}}%
\newcommand{\hlstr}[1]{\textcolor[rgb]{0.192,0.494,0.8}{#1}}%
\newcommand{\hlcom}[1]{\textcolor[rgb]{0.678,0.584,0.686}{\textit{#1}}}%
\newcommand{\hlopt}[1]{\textcolor[rgb]{0,0,0}{#1}}%
\newcommand{\hlstd}[1]{\textcolor[rgb]{0.345,0.345,0.345}{#1}}%
\newcommand{\hlkwa}[1]{\textcolor[rgb]{0.161,0.373,0.58}{\textbf{#1}}}%
\newcommand{\hlkwb}[1]{\textcolor[rgb]{0.69,0.353,0.396}{#1}}%
\newcommand{\hlkwc}[1]{\textcolor[rgb]{0.333,0.667,0.333}{#1}}%
\newcommand{\hlkwd}[1]{\textcolor[rgb]{0.737,0.353,0.396}{\textbf{#1}}}%
\let\hlipl\hlkwb

\usepackage{framed}
\makeatletter
\newenvironment{kframe}{%
 \def\at@end@of@kframe{}%
 \ifinner\ifhmode%
  \def\at@end@of@kframe{\end{minipage}}%
  \begin{minipage}{\columnwidth}%
 \fi\fi%
 \def\FrameCommand##1{\hskip\@totalleftmargin \hskip-\fboxsep
 \colorbox{shadecolor}{##1}\hskip-\fboxsep
     % There is no \\@totalrightmargin, so:
     \hskip-\linewidth \hskip-\@totalleftmargin \hskip\columnwidth}%
 \MakeFramed {\advance\hsize-\width
   \@totalleftmargin\z@ \linewidth\hsize
   \@setminipage}}%
 {\par\unskip\endMakeFramed%
 \at@end@of@kframe}
\makeatother

\definecolor{shadecolor}{rgb}{.97, .97, .97}
\definecolor{messagecolor}{rgb}{0, 0, 0}
\definecolor{warningcolor}{rgb}{1, 0, 1}
\definecolor{errorcolor}{rgb}{1, 0, 0}
\newenvironment{knitrout}{}{} % an empty environment to be redefined in TeX

\usepackage{alltt}
\usepackage{amsmath,amssymb,float,amsfonts,enumitem,tcolorbox,hyperref}

%!TEX root = EM_singular_models.tex

% PDF margin etc settings
\setlength{\textwidth}{\paperwidth}
\addtolength{\textwidth}{-6cm}
\setlength{\textheight}{\paperheight}
\addtolength{\textheight}{-4cm}
\addtolength{\textheight}{-1.1\headheight}
\addtolength{\textheight}{-\headsep}
\addtolength{\textheight}{-\footskip}
\setlength{\oddsidemargin}{0.5cm}
\setlength{\evensidemargin}{0.5cm}

\long\def\comment#1{}
\definecolor{battleshipgrey}{rgb}{0.52, 0.52, 0.51}
\definecolor{darkgray}{rgb}{0.66, 0.66, 0.66}
\definecolor{darkgreen}{rgb}{0.0, 0.2, 0.13}
\definecolor{darkspringgreen}{rgb}{0.09, 0.45, 0.27}
\definecolor{dukeblue}{rgb}{0.0, 0.0, 0.61}
\definecolor{olivedrab7}{rgb}{0.24, 0.2, 0.12}
\definecolor{darkblue}{rgb}{0.0, 0.0, 0.55}
\definecolor{darkscarlet}{rgb}{0.34, 0.01, 0.1}
\definecolor{candyapplered}{rgb}{1.0, 0.03, 0.0}
\definecolor{ao(english)}{rgb}{0.0, 0.5, 0.0}
\definecolor{applegreen}{rgb}{0.55, 0.71, 0.0}
\definecolor{orange}{rgb}{1.0, 0.65, 0.0}

\newcommand{\sol}[1]{{\bf{{\blue{{\\ \textbf{Ans.} #1}}}}}}

\newcommand{\red}[1]{\textcolor{red}{#1}}
\newcommand{\blue}[1]{\textcolor{blue}{#1}}
\newcommand{\green}[1]{\textcolor{green}{#1}}
\newcommand{\orange}[1]{\textcolor{orange}{#1}}
\newcommand{\highlight}[1]{\textcolor{applegreen}{#1}}

\newcommand{\rdcomment}[1]{{\bf{{\red{{RD --- #1}}}}}}
\newcommand{\yccomment}[1]{{\bf{{\orange{{YC --- #1}}}}}}

\renewcommand{\vec}[1]{\mathbf{#1}}
\newcommand{\mat}[1]{\mathbf{#1}}

% Basic Math notations
% Dimension etc.
\newcommand{\dims}{\ensuremath{d}}
\newcommand{\real}{\ensuremath{\mathbb{R}}}
\newcommand{\naturalnum}{\ensuremath{\mathbb{N}}}
\newcommand{\Ind}{\ensuremath{\mathbb{I}}}
% Probability
\newcommand{\borel}{\ensuremath{\mathcal{B}}}
\newcommand{\lebesgue}{\ensuremath{\mathfrak{L}}}
\newcommand{\powerset}{\ensuremath{\mathfrak{P}}}
\newcommand{\lebesguemeasure}{\ensuremath{\lambda}}
\newcommand{\ball}{\ensuremath{\mathbb{B}}}
\newcommand{\Exs}{\ensuremath{{\mathbb{E}}}}
\newcommand{\Prob}{\ensuremath{{\mathbb{P}}}}
\newcommand{\Law}{\mathcal{L}}
\newcommand{\Normal}{\ensuremath{\mathcal{N}}}
\newcommand{\Ber}{\ensuremath{\mbox{Ber}}}
\newcommand{\Ent}{\text{Ent}}
\newcommand{\Var}{\text{Var}}


% Brackets
\newcommand{\brackets}[1]{\left[ #1 \right]}
\newcommand{\parenth}[1]{\left( #1 \right)}
\newcommand{\bigparenth}[1]{\big( #1 \big)}
\newcommand{\biggparenth}[1]{\bigg( #1 \bigg)}
\newcommand{\braces}[1]{\left\{ #1 \right \}}
\newcommand{\abss}[1]{\left| #1 \right |}
\newcommand{\angles}[1]{\left\langle #1 \right \rangle}
\newcommand{\ceils}[1]{\left\lceil #1 \right \rceil}
\newcommand{\floors}[1]{\left\lfloor #1 \right \rfloor}
\newcommand{\tp}{^\top}

% Some vector/matrix norms
\newcommand{\matnorm}[3]{|\!|\!| #1 | \! | \!|_{{#2}, {#3}}}
\newcommand{\matsnorm}[2]{\left|\!\left|\!\left| #1 \right|\!\right|\!\right|_{{#2}}}
\newcommand{\vecnorm}[2]{\left\| #1\right\|_{#2}}




\title{STAT 154: Project 2 Cloud Data}
\author{Release date: \textbf{Wednesday, April 10}}
\date{Due by: \textbf{11 PM, Wednesday, May 1}}
\IfFileExists{upquote.sty}{\usepackage{upquote}}{}
\begin{document}

\maketitle

\section*{Please read carefully!}
\begin{itemize}
  \item It is a good idea to revisit your notes, slides and reading;
and synthesize their main points BEFORE doing the project.
  \item \emph{For this project, we adapt a zero tolerance policy with 
  incorrect/late submissions (no emails please) to Gradescope.}
  \item The recommended work of this project is at least 20 hours (at least 10 hours / person). Plan ahead and start early. 
  \item We need two things:
  \begin{enumerate}[label=(\alph*)]
    \item A main pdf report \textbf{(font size at least 11 pt, less or equal
    to 12 pages)} generated by Latex, Rnw or Word is required to be
    submitted to Gradescope.
    \begin{itemize}
      \item Provide top class (research-paper level) writing, useful
    well-labeled figures and no code in this pdf. Arrange text and figures
    compactly (.Rnw may not be very useful for this).
    \item You can choose a title for the report and a team name as per your
    liking (\emph{get creative!}). Do provide the names and student ID of
    your teammates below the title.
    \item Your report should conclude with an acknowledgment section, where
    you provide brief discussion about the contributions of each member,
    \textbf{and} the resources you used, credit all the help you took
    and briefly outline the way you proceeded with the project.
    \end{itemize}
    \item A link to your GitHub Repo at the end of your write-up that contains
    all your code (see Section 5 for more details).
  \end{enumerate}
  \item \textbf{Be visual \emph{and} quantitative:} Remember projects are graded differently when compared to homework---one line answer without explanation is usually not enough. Make your findings succinct and try to convince us with good arguments supported by numbers and figures.
Putting yourself in reader's shoes and reading the report out loud usually helps. The standards for grading are \emph{very high} this time. We will be very picky with figures: Lack of proper titles and axis labels will lead to loss of several points.
  
\end{itemize}

\newpage

\section*{Overview of the project} % (fold)
\label{sec:overview_of_the_project}

The goal of this project is the exploration and modeling of cloud detection in 
the polar regions based on radiance recorded automatically by the MISR sensor 
abroad the NASA satellite Terra. You will attempt to build a classification 
model to distinguish the presence of cloud from the absence of clouds in
the images using the available signals/features. Your dataset has ``expert
labels'' that can be used to train your models. When you evaluate your
results, imagine that your models will be used to distinguish
clouds from non-clouds on a large number of images that won't have these 
``expert'' labels.

On Piazza, you will find a zip archive with three files: \textbf{image1.txt},
\textbf{image2.txt}, \textbf{image3.txt}. Each contains one picture from
the satellite. Each of these files contains several rows each with 11 columns 
described in the Table~\ref{tab:feature} below. All five radiance angles
are raw features, while NDAI, SD, and CORR are features that are computed
based on subject matter knowledge. More information about the features is
in the article \textbf{yu2008.pdf}. The sensor data is multi-angle and recorded
in the red-band. 
For more information about MISR, see \textbf{http://www-misr.jpl.nasa.gov/}.

\begin{table}[h]
\centering
\begin{tabular}{|c|c|}
\hline
 01 & y coordinate \\ \hline
 02 & x coordinate \\ \hline
 03 & expert label (+1 = cloud, -1 = not cloud, 0 unlabeled)\\ \hline
 04 & NDAI \\ \hline
 05 & SD \\ \hline
 06 & CORR \\ \hline
 07 & Radiance angle DF\\ \hline
 08 & Radiance angle CF\\ \hline
 09 & Radiance angle BF\\ \hline
 10 & Radiance angle AF\\ \hline
 11 & Radiance angle AN\\ \hline
\end{tabular}
\label{tab:feature}
\caption{Features in the cloud data.}
\end{table}

\section{Data Collection and Exploration (30 pts)}
\begin{enumerate}[label=(\alph*)]
\item \textbf{Write a half-page summary} of the paper, including at least
the purpose of the study, the data, the collection method, its conclusions
and potential impact.
\item \textbf{Summarize} the data, i.e., $\%$ of pixels for the different
classes. \textbf{Plot well-labeled beautiful maps} using $x, y$ coordinates
the expert labels with color of the region based on the expert labels.
\textbf{Do you observe some trend/pattern? Is an i.i.d. assumption for
the samples justified for this dataset?}
\item \textbf{Perform a visual and quantitative EDA} of the dataset, e.g.,
summarizing (i) pairwise relationship between the features themselves and
(ii) the relationship between the expert labels with the individual features.
\textbf{Do you notice differences} between the two classes (cloud, no cloud)
based on the radiance or other features (CORR, NDAI, SD)?
\end{enumerate}

\section{Preparation (40 pts)}
Now that we have done EDA with the data, we now prepare to train our model.
\begin{enumerate}[label=(\alph*)]
\item (Data Split) \textbf{Split the entire data} (image1.txt, image2.txt,
image3.txt) into three sets: training,  validation and test. Think carefully
about how to split the data. \textbf{Suggest at least two non-trivial different
ways} of splitting the data which takes into account that the data is not i.i.d.
\item (Baseline) \textbf{Report the accuracy of a trivial classifier} which
sets all labels to -1 (cloud-free) on the validation set and on the test set. 
In what scenarios will such a classifier have high average accuracy?
\emph{Hint: Such a step provides a baseline to ensure that the classification
problems at hand is not trivial.}
\item (First order importance) Assuming the expert labels as the
truth, and without using fancy classification methods, suggest
three of the ``best'' features, \textbf{using quantitative and visual justification}. Define your ``best'' feature criteria clearly. Only the relevant plots are necessary. Be sure to give this careful consideration, as it relates to subsequent problems.
\item Write a generic cross validation (CV) function \textbf{CVgeneric} in R that takes a generic classifier, training features, training labels, number of folds $K$ and a loss function (at least classification accuracy should be there) as inputs and outputs the $K$-fold CV loss on the training set.  Please remember to put it in your github folder in Section 5.
\end{enumerate}

\section{Modeling (40 pts)}
We now try to fit different classification models and assess the fitted
models using different criterion. For the next three parts, we expect you
to try \emph{logistic regression and at least three other methods}.
\begin{enumerate}[label=(\alph*)]
\item \textbf{Try several classification methods and assess their fit using
cross-validation (CV). Provide a commentary on the assumptions for the
methods you tried and if they are satisfied in this case.} 
Since CV does not have a validation set, you can merge your training and
validation set to fit your CV model. 
\textbf{Report} the accuracies across
folds (and not just the average across folds) and the test accuracy. CV-results
for both the ways of creating folds (as answered in part 2(a)) should be
reported. Provide a brief commentary on the results. Make sure you honestly
mention all the classification methods you have  tried.
\item \textbf{Use ROC curves to compare the different methods.} 
Choose a cutoff value and highlight it on the ROC curve. Explain your choice
of the cutoff value. 
\item (Bonus) Assess the fit using other relevant metrics.
\end{enumerate}

\section{Diagnostics (50 pts)}
\emph{Disclaimer:} The questions in this section are open-ended.
Be visual and quantitative! The gold standard arguments would be able to
convince National Aeronautics and  Space Administration (NASA) to use your
classification method---in which case Bonus points will be awarded.
\begin{enumerate}[label=(\alph*)]
\item Do an in-depth analysis of a good classification model
of your choice by showing some diagnostic plots or information related to
convergence or parameter estimation.
\item For your best classification model(s), do you notice any patterns in the 
misclassification errors? Again, use quantitative and visual methods of analysis. Do you notice problems in particular regions, or in
specific ranges of feature values?
\item Based on parts 4(a) and 4(b), can you think of a better classifier?
How well do you think your model will work on future data without expert 
labels?
\item Do your results in parts 4(a) and 4(b) change as you modify the
way of splitting the data?
\item Write a paragraph for your conclusion.
\end{enumerate}

\section{Reproducibility (10 pts)}
In addition to a writeup of the above results, please provide a one-line link 
to a public GitHub repository containing everything necessary to reproduce
your writeup. Specifically, imagine that at some point an error is discovered
in the three image files, and a future researcher wants to check whether
your results hold up with the new, corrected image files. This researcher
should be able to easily re-run all your code and produce all your figures
and tables. This repository should contain:
\begin{enumerate}[label=(\roman*)]
  \item The pdf of the report,
  \item the raw Latex, Rnw or Word used to generate your report,
  \item your R code (with CVgeneric function in a separate R file),
  \item a README file describing, in detail, how to reproduce your paper
  from scratch (assume researcher has access to the images).
\end{enumerate}
You might want to take a look at the GitHub's tutorials 
\href{https://guides.github.com/}{https://guides.github.com/}.

\section*{Final remarks} % (fold)
\label{sec:final_remarks}
% As a reminder:
\begin{itemize}
  \item Make sure to read the instructions for the submission on Page
  1.
  \item Note that we will enforce a \textbf{zero tolerance policy for last
  minute / late requests (no emails please) this time.} Start early and
  plan ahead. If something
  is falling apart or not working, see us in office hours.
\end{itemize}

\end{document}
